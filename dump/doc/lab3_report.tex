\title{Lab 3}
\author{
        Kebur Fantahun \\
                Department of Physics\\
        TTU---Texas Tech University\\
        Lubbock, Texas, \underline{U.S.A.}
}

\date{\today}

\documentclass[12pt]{article}

\usepackage{pdfpages}

\begin{document}

\maketitle

\section{Part 1. Precision of Numerical Differentiation}
The slope of the plots inside each region agrees with my expectations. The region of decreasing slope is where the round-off error dominates. The region of increasing slope is where the approximation error dominates. At first the slope did not agree but after correcting the spacing of the range of h the slopes began to look like $\mathcal{O}(h)$ and $\mathcal{O}(h^2)$ respectively. Section~\ref{plots} depicts the median relative error plots from attempting to take three first derivatives and one second derivative of the functions $e^x$, $ln(x+1)$, and $sin(2x)$ .

\section{Plots}\label{plots}
Please find the plots on page 2. \\
\includepdfmerge[nup=2x2]{fig1.pdf,  fig2.pdf, fig3.pdf}

\section{Part 2. Optimizing Projectile Firing Angle}

\section{Choices}\label{choices}
In this section, I will describe the choices for certain parameters and reasons for those choices. \\
I chose the Runge-Kutta RK45 algorithm since it allowed me to ensure that the flight distance as a function of the firing angle has a continouous second derivative. Parabolic interpolations, i.e. Brent's method, are not useful unless that second derivative is constant hence the choice of Runge-Kutta RK45.

\end{document}
This is never printed
