\title{Lab 4}
\author{
        Kebur Fantahun \\
                Department of Physics\\
        TTU---Texas Tech University\\
        Lubbock, Texas, \underline{U.S.A.}
}

\date{\today}

\documentclass[12pt]{article}

\usepackage{pdfpages}

\begin{document}

\maketitle

\section{Part 1. }
Gaussian quadrature converges faster than trapezoidal and Simpsons rule integration approximations, at least for the functions exp(x) and log(x). For the sine integral approximation, the opposite is true, that is Gaussian quadrature is beaten by the Simpsons and trapezoidal rule. I have not figured out the weird behavior of my sin plot but there is a spike, in the Simpsons rule line, around N = 10 for some reason. Simpsons rule tries to approximate the equation of a quadratic. What this means for us is that when we use the simpsons rule on sine, we see really good results because the peaks of the sine function are curves. 
\section{Plots}\label{plots}
Please find the plots on page 2. \\
\includepdfmerge[nup=2x2]{exp.pdf,  log.pdf, sin.pdf}

\section{Part 2. }
I spent alot of my time working on part 1 this time, I will have part 2 finished on the next lab.

\end{document}
This is never printed
